\chapter{Antecedentes}
\label{sec-2}

\section{Twitter}
\label{sec-2.1}

Twitter es una red social online que permite conectar a
personas mediante la comunicación de mensajes cortos, rápidos y frecuentes. Estos
mensajes son publicados en el perfil del usuario que los emite, pueden
ser vistos directamente por los seguidores de este usuario o ser
vistos directamente en el perfil o buscándolos mediante una
funcionalidad que provee el servicio. Además, un usuario puede
\emph{seguir} a otros para poder ver en su \emph{timeline} los mensajes de todos
a quienes sigue.

FIGURA TWITTER

Estos mensajes, o \emph{tweets}, pueden además \emph{mencionar} a otros
usuarios, mediante la convención ``=@usuario [texto]='' indica que
se está mencionando a la persona con el nombre ``usuario''. Adicionalmente,
existen varias convenciones o costumbres que han surgido a lo largo
del tiempo en esta red desde sus inicios el año 2007:


\begin{itemize}
\item Respuestas o \emph{replies}: son mensajes del tipo \texttt{@usuario [texto]},
  que ocurren usualmente en una conversación entre dos usuarios.
\item Menciones o \emph{mentions}: un poco más general a una respuesta, el
  nombre del usuario mencionado puede estar en cualquier parte del
  mensaje. La diferencia semántica es que no se le habla
  ``directamente'' al usuario mencionado, como en una respuesta, sino
  que sólo es mencionado por si el mensaje es de su interés o no.
\item \emph{Retweets}: son mensajes del tipo \texttt{RT @usuario: [texto]}. Ocurren
  cuando se quiere compartir el mensaje de otro usuario, o citarlo
  para mencionarlo en el mismo mensaje.
\item \emph{Hashtags}: son palabras precedidas por el caracter \#, que indican
  un identificador a cierto evento o suceso dentro o fuera de la
  red. Suelen usarse para categorizar de cierta forma un tópico, pero
  son libres de usarse como los usuarios quieran.
\item Mensaje simple: un mensaje sin menciones ni hashtags.
\end{itemize}
Ejemplos:

\begin{itemize}
\item Mensaje simple: \texttt{Jason Funk disipa patitos};
\item Respuesta: \texttt{@jason estoy de acuerdo con lo que dices};
\item Mención: \texttt{creo que @jason es una cumbre de sabiduría};
\item Retweet: \texttt{RT @jason: Jason Funk disipa patitos}; y
\item Hashtag: \texttt{Estoy escribiendo mi memoria \#dcc \#summarization}
\end{itemize}
Estos mensajes están limitados a 140 caracteres de extensión. Sumando
esto a la integración de la red con otros servicios y dispositivos, y
a la cantidad de mensajes publicados cada minuto, permite utilizar
esta red como una gran fuente de datos.

Twitter además provee varios servicios adicionales, como por ejemplo,
un servicio de acortamiento de URLs, para permitir incluir una URL
larga sin perjudicar la cantidad de caracteres restantes para el
mensaje; un servicio de alojamiento de fotos y vídeos, para hacer más
sencilla la publicación de mensajes multimedia desde dispositivos
móviles; un servicio de búsqueda que permite buscar una cantidad
determinada de tweets sobre un término de búsqueda o un hashtag.


\section{Identificación automática de eventos}
\label{sec-2.2}

\section{Clustering de documentos}
\label{sec-2.3}

\subsection{Evaluación de clusterings}
\label{sec-2.3.1}

\section{Resúmenes automáticos}
\label{sec-2.4}

\subsection{Evaluación de resúmenes}
\label{sec-2.4.1}

\section{Ranking de documentos}
\label{sec-2.5}
