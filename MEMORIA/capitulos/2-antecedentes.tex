\chapter{Antecedentes}
\label{sec-1}


  Para poder describir correctamente tanto el problema como la
  solución implementada, es necesario dar los punteros y conceptos
  básicos que los involucran. En este capítulo se discutirán los
  siguientes tópicos:

\begin{itemize}
\item La red social Twitter, la cual es utilizada como fuente de datos y
    documentos para este trabajo.
\item Clustering de documentos, y en general, estrategias de clustering,
    las cuales tienen muchas aplicaciones prácticas. En este trabajo
    fue utilizada una de estas estrategias para poder determinar los
    subtópicos de cada evento.
\item La identificación automática de eventos, la cual, si bien se
    utilizó un enfoque más simple para este trabajo, sirve para
    indicar en qué aspectos es posible extender este trabajo en el
    futuro.
\item Resúmenes automáticos: una sucinta definición, y algunos enfoques
    que han existido en el tiempo para este procedimiento.
\item Ranking de documentos, o cómo generar órdenes de acuerdo a
    relevancia.
\end{itemize}
  Casi todos estos tópicos, a excepción del primero, involucran
  técnicas de Minería de Datos, Recuperación de la Información y
  Aprendizaje de Máquinas, entre otras áreas.

\section{Twitter}
\label{sec-1.1}

   Twitter es una red social online que permite conectar a
   personas mediante la comunicación de mensajes cortos, rápidos y   frecuentes\footnote{\href{https://support.twitter.com/groups/31-twitter-basics/topics/104-welcome-to-twitter-support/articles/13920-get-to-know-twitter-new-user-faq}{https://support.twitter.com/groups/31-twitter-basics/topics/104-welcome-to-twitter-support/articles/13920-get-to-know-twitter-new-user-faq} }. Estos
   mensajes son publicados en el perfil del usuario que los emite,
   pueden ser vistos directamente por los seguidores de este usuario o
   ser vistos directamente en el perfil o buscándolos mediante una
   funcionalidad que provee el servicio. Además, un usuario puede
   \emph{seguir} a otros para poder ver en su \emph{timeline} o perfil privado
   los mensajes de todos a quienes sigue.

\begin{figure}[h!b]
  \centering
  \includegraphics[width=12cm]{./img/twitter.png}
  \caption[Timeline de Twitter.]
   {Timeline de Twitter. En éste se ve una lista en orden cronológico
  de los tweets generados por los usuarios que sigue el usuario
  actual. Además, el sitio incluye tweets promocionados por cuentas
  que pagan por dicho servicio, como es el caso del primer tweet en la
  lista.}
\end{figure}

   Estos mensajes, o \emph{tweets}, sólo son cadenas de caracteres con
   metadatos que el mismo servicio asigna una vez enviado a la red
   social. Desde sus inicios (año 2007) se han añadido algunas capacidades
   adicionales a estos mensajes, como la de poner URLs, imágenes,
   vídeos, etc. Además, existen varias convenciones que han surgido a
   lo largo del tiempo. A continuación se describe una lista de tipos
   de mensajes que existen en Twitter, originados por estas convenciones:

\begin{enumerate}
\item Respuestas o \emph{replies}: son mensajes del tipo \texttt{@usuario [texto]},
      que ocurren usualmente en una conversación entre dos usuarios.
\item Menciones o \emph{mentions}: un poco más general a una respuesta, el
      nombre del usuario mencionado puede estar en cualquier parte del
      mensaje. La diferencia semántica es que no se le habla
      ``directamente'' al usuario mencionado, como en una respuesta, sino
      que sólo es mencionado por si el mensaje es de su interés o no.
\item \emph{Retweets}: son mensajes del tipo \texttt{RT @usuario: [texto]}. Ocurren
      cuando se quiere compartir el mensaje de otro usuario, o citarlo
      para mencionarlo en el mismo mensaje.
\item \emph{Hashtags}: son palabras precedidas por el caracter \#, que indican
      un identificador a cierto evento o suceso dentro o fuera de la
      red. Suelen usarse para categorizar de cierta forma un tópico, pero
      son libres de usarse como los usuarios quieran.
\item Mensaje simple: un mensaje sin menciones ni hashtags.
\end{enumerate}
  Ejemplos:

\begin{itemize}
\item Mensaje simple: \texttt{Jason Funk disipa patitos};
\item Respuesta: \texttt{@jason estoy de acuerdo con lo que dices};
\item Mención: \texttt{creo que @jason es una cumbre de sabiduría};
\item Retweet: \texttt{RT @jason: Jason Funk disipa patitos}; y
\item Hashtag: \texttt{Estoy escribiendo mi memoria \#dcc \#summarization}
\end{itemize}
  Estos mensajes están limitados a 140 caracteres de extensión. Sumando
  esto a la integración de la red con otros servicios y dispositivos, y
  a la cantidad de mensajes publicados cada minuto, permite utilizar
  esta red como una gran fuente de datos.

  Twitter además provee varios servicios adicionales, como por ejemplo,
  un servicio de acortamiento de URLs, para permitir incluir una URL
  larga sin perjudicar la cantidad de caracteres restantes para el
  mensaje; un servicio de alojamiento de fotos y vídeos, para hacer más
  sencilla la publicación de mensajes multimedia desde dispositivos
  móviles; un servicio de búsqueda que permite buscar una cantidad
  determinada de tweets sobre un término de búsqueda o un hashtag,
  entre otros servicios.







\section{Clustering de documentos}
\label{sec-1.2}


   El análisis de clusters o clustering es el proceso de encontrar
   grupos de objetos, tal que los objetos en un grupo sean similares
   entre sí (o relacionados) y que sean diferentes (o no relacionados)
   a los objetos de otros grupos. Algunas aplicaciones del análisis de
   clusters son, entre otras:
\begin{itemize}
\item Encontrar clusters naturales y describir sus propiedades (\emph{data understanding});
\item Encontrar agrupamientos útiles (\emph{data class identification});
\item Encontrar representantes de grupos homogéneos (\emph{data reduction});
\item Encontrar perturbaciones aleatorias de los datos (\emph{noise detection});
\item Encontrar objetos inusuales (\emph{outliers detection});
\item etc.
\end{itemize}
   Se denomina cluster a un grupo de objetos, mientras que
   clustering puede referirse al conjunto de clusters o al proceso de
   encontrarlos. Existen diversos tipos de procesos de clustering, una
   de las distinciones más importantes es entre los clusters
   jerárquicos y los particionales:
\begin{itemize}
\item Un clustering jerárquico es un conjunto de clusters anidados,
     organizados más bien como un árbol. Cortando el árbol en
     cualquier nivel da como resultado un clustering potencialmente
     distinto.
\item El clustering particional es un conjunto de clusters de forma de
     partición del conjunto total, es decir, cada objeto está
     contenido en un sólo subconjunto o cluster.
\end{itemize}
   Para describir el proceso aplicado a documentos, primero se
   describirán los modelos de representación más importantes para,
   de forma de definir la noción de documento y luego los
   algoritmos de clustering aplicados a éstos.

\subsection{Modelos de representación de documentos}
\label{sec-1.2.1}


    \subsubsection{Standard Boolean Model}

    El modelo booleano es un modelo de representación de
    documentos. En él, los documentos son vectores de \emph{términos}:

    $$d = (w_1,w_2,\ldots,w_m)$$

    Donde un término es un $n$-grama del texto del documento.

    \begin{defn} Un $n$-grama es una secuencia contigua de $n$ ítems a
    partir de un texto. \end{defn}

    La definición de ítem dependerá de la aplicación: en lenguaje
    natural el texto a su vez dependerá del idioma, por ejemplo, si el
    texto está en inglés o en japonés, la distinción entre palabras
    es distinta para cada uno.

    No existe una medida de ``similitud'' como tal en este modelo, sino
    que se considera el calce exacto entre los términos de una query
    $q$ y un documento $d$. La query puede ser una consulta hecha por
    un usuario al conjunto de documentos, o bien un documento del
    mismo conjunto.

    Una consulta es una fórmula de lógica proposicional que pide los
    documentos que contengan o no ciertos términos.

    \subsubsection{Bag of words Model}

    En el modelo Bag of Words un documento $d$ es representado como un
    conjunto de pares $(w_i, f_i)$, $i\in[1..m']$, donde $w_i$ es un
    término del documento, $f_i$ es la frecuencia de $w_i$ en el
    mismo, y $m'$ es la cantidad de términos distintos en el
    documento.

    La ventaja principal por sobre el modelo anterior es que permite
    hacer calces parciales entre consultas y documentos. Este modelo
    es comúnmente utilizado para hacer clasificación de documentos,
    por ejemplo, para determinar si un correo electrónico es o no
    spam.

    \subsubsection{Vector Space Model}

    El \emph{Vector Space Model} es un modelo un poco más general que el
    anterior. Un documento $d$ es representado como un vector de pesos
    asociados a los términos:

    $$d = (f(w_1), f(w_2), \ldots, f(w_m))$$

    Cada dimensión de este vector corresponde al peso asociado a un
    término del documento.

    El peso puede ser directamente la frecuencia del término dentro
    del documento:

    $$\freq(w,d) = |\{w : w \in d\}|$$

    O bien, normalizar esta frecuencia para evitar que documentos más
    largos sean más relevantes sólo por su extensión:

    $$\tf_0(w,d) = \left\{
    \begin{array}{l l}
    1 & \quad \textrm{si $w \in d$}\\
    0 & \quad \textrm{si no}\\
    \end{array} \right.$$

    $\tf_0$ o \emph{Term Frequency} es una primera aproximación a medir la
    frecuencia de un término en un documento. Sin embargo, esta nueva
    aproximación sufre de la desventaja de que ahora un documento con
    una ocurrencia del término será igual de relevante que algún
    documento que mencione varias veces el término (por ejemplo, un
    diccionario que tiene el término una vez contra un artículo sobre
    el tema). Otra alternativa, considera no castigar demasiado a los
    documentos con pocas ocurrencias, pero tampoco beneficiar mucho a
    los que tengan muchas:

    $$\tf_1(w,d) = 1 + \log(\freq(w,d))$$

    La solución más utilizada considera la proporción con respecto al
    término con más ocurrencias, para esto, se normaliza por el tamaño
    del documento:

    $$\tf(w,d) = \frac{\freq(w,d)}{max\{\freq(t,d) : t \in d\}}$$

    Otro problema que tiene utilizar esta medida como los pesos de los
    términos, es que un término muy repetido entre todos los
    documentos que hablan de un mismo tema puede significar que no es
    muy relevante (por ejemplo, las \emph{stopwords} o palabras vacías, son
    por lo general las preposiciones, artículos, pronombres,
    etc.). Para esto, se considera además ponderar por el inverso de
    la frecuencia entre los documentos; es decir, un término frecuente
    entre todos los documentos ve su peso castigado a diferencia de un
    término que sólo es mencionado una vez en un documento. Esta
    medida es llamada \emph{Inverse Document Frequency} o $\idf$:

    $$\idf(t, D) = \log \frac{ |D| } {1 + |\{d \in D : t \in d\}| }$$

    Finalmente, el peso de un término es la ponderación de su
    frecuencia dentro del documento con el inverso de la frecuencia
    entre los documentos, o $\tfidf$:

    $$\tfidf(t,d,D) = \tf(t,d) \times \idf(t,D)$$

    Para el resto de este trabajo se considerará la representación de
    documentos en este modelo.

\subsection{Medidas de similitud}
\label{sec-1.2.2}


    A lo largo de los años, dos formas han sido las usuales para
    determinar similitud entre
    documentos\cite{Zhao02criterionfunctions}: calculando el coseno de
    los dos vectores, y la distancia euclidiana.

    El coseno de dos documentos $d_i$ y $d_j$ se define como

    $$\cos(d_i, d_j) = \frac{d_i^td_j}{\|d_i\|\|d_j\|}$$

    Si ambos $d_i$ y $d_j$ son iguales, entonces la fórmula anterior
    se evalúa a 1, y 0 si no tienen nada en común, es decir, ambos
    vectores son ortogonales.

    La distancia euclidiana entre $d_i$ y $d_j$ se define como

    $$d(d_i,d_j) = \sqrt{(d_i-d_j)^t(d_i-d_j)} = \|d_i-d_j\|$$

    En este caso, si la distancia es 0, entonces ambos documentos son
    idénticos. Y si ambos son ortogonales, la distancia será
    $\sqrt{2}$.

    Si ambos documentos están normalizados (su norma es 1), entonces
    ambas medidas serán muy parecidas, aun así siendo una de similitud
    y la otra de distancia.

\subsection{Algoritmos de clustering}
\label{sec-1.2.3}


    Se considerará la siguiente definición para el problema de
    clustering:

    Dado un entero $k$ y un conjunto de $n$ puntos en $\mathbb{R}^v$,
    el objetivo es determinar $k$ puntos tal que se minimice $\phi$,
    la suma de las distancias al cuadrado de cada punto y su centro
    más cercano.

    Este problema es NP-hard, sin embargo, existe un algoritmo que
    permite realizar una búsqueda local y determinar $k$ centros
    en un tiempo razonable\footnote{Lloyd, S.; , ``Least squares quantization in PCM,'' Information Theory, IEEE Transactions on , vol.28, no.2, pp. 129- 137, Mar 1982 }.

    K-means es un algoritmo de clustering particional, en el cual cada
    cluster tiene un \emph{centroide} asociado, esto es, típicamente, un punto
    el cual es el promedio de todos los puntos del cluster, y no
    necesariamente corresponde a un dato real.

    El algoritmo genera $k$ clusters, donde $k$ es un parámetro del
    algoritmo, tal que cada punto pertence al cluster cuyo centroide es el
    más cercano a éste.

\begin{algorithm}[H]
 \KwData{$k>0$, conjunto $D$ de puntos}
 \KwResult{Asignación de cada punto a un cluster $i \in [1..k]$}
 Seleccionar $k$ puntos como centroides iniciales\;
 \While{Los centroides cambien a cada iteración}{
  Formar $k$ clusters asignando todos los puntos al centroide más cercano\;
  Recalcular los centroides de cada cluster\;
 }
 \caption{K-means}
\end{algorithm}

    Los clusters generados dependerán de la elección inicial de los
    centroides, y usualmente basta con pocas iteraciones para la
    convergencia. La complejidad de este algoritmo es $O(nkIv)$, donde $n$
    es el número de puntos, $k$ el parámetro de la cantidad de clusters,
    $I$ es la cantidad de iteraciones que hará el algoritmo y $v$ es la
    cantidad de dimensiones de los vectores.

    K-means también puede verse desde el enfoque de optimizar una
    función criterio. La medida más común es la
    \emph{suma del error cuadrado} o Sum of Squared Error (SSE). Para cada
    punto, su \emph{error} es la distancia al cluster más cercano:
    $$SSE = \sum_{i=1}^k\sum_{x\in C_i} \dist^2(m_i,x)$$
    Donde $x$ es un punto en el cluster $C_i$, y $m_i$ es su
    centroide. Dados dos clusters, en cada paso se escoge el que tenga
    menor error.
    Un problema importante de K-means es escoger los centroides
    iniciales. Dado que la probabilidad de acertar a los centroides
    ``reales'' (precisión) es muy baja, es posible que un clustering
    tenga un mal error cuadrático, de hecho, el radio competitivo con
    la solución óptima es no acotado incluso para $k$ y $n$ fijos.
    Existen varias formas de evitar
    este problema, por ejemplo, ejecutando varias veces K-means,
    usar clustering jerárquico para determinar los centroides
    iniciales, determinar más de $k$ centroides y luego elegir los $k$
    más separados, etc.

    Un enfoque utilizado es el de usar el algoritmo
    K-means++\cite{Arthur:2007:KAC:1283383.1283494}, el cual tiene
    como objetivo determinar los centroides iniciales para
    K-Means. Este algoritmo garantiza un radio competitivo de
    $O(\log k)$ con respecto al error cuadrático esperado del óptimo.
    El algoritmo K-means++ escoge los centroides con cierta
    probabilidad que depende de la distancia al centroide más cercano,
    lo cual garantiza una buena elección inicial, y luego continúa con
    el algoritmo usual. Sea $D(x)$ la distancia de $x$ al centroide
    más cercano. El algoritmo es como sigue:

\begin{algorithm}[H]
 \KwData{$k>0$, conjunto $D$ de puntos}
 \KwResult{Asignación de cada punto a un cluster $i \in [1..k]$}
Escoger un centroide $c_1$ al azar uniformemente de $D$\;
Escoger un nuevo centroide $c_i$, escogiendo $x \in D$ con probabilidad $\frac{D(x)^2}{\sum_{x' \in D}D(x')^2}$\;
Repetir paso 1 hasta haber escogido $k$ centros\;
Proceder con K-means estándar\;
 \caption{K-means++}
\end{algorithm}


\subsection{Evaluación de clusters}
\label{sec-1.2.4}


    Existen dos enfoques para evaluar clusterings: validación interna
    y externa\footnote{\href{http://en.wikipedia.org/wiki/Cluster\_analysis\#Evaluation\_of\_clustering\_results}{http://en.wikipedia.org/wiki/Cluster\_analysis\#Evaluation\_of\_clustering\_results} } La validación interna considera evaluar el clustering
    con respecto a los datos que han sido agrupados (es decir, sin
    información adicional), mientras que la evaluación externa
    considera datos adicionales, como datos pre-clasificados (conjunto
    de prueba, de entrenamiento, etc), o los clusters ``reales''.

    Por otra parte, dos de las métricas más usadas para evaluar
    clusterings, segun \cite{Zhao02criterionfunctions} son:

\begin{itemize}
\item \textbf{Entropía}: mide cómo son distribuidas las clases de documentos
      entre cada cluster. Se define formalmente como

      $$E(C_r) = -\frac{1}{\log q}\sum_{i=1}^q\frac{n^i_r}{n_r}\log\frac{n^i_r}{n_r}$$

      donde $q$ es el número de clases en el dataset, y $n^i_r$ es el
      número de documentos de la $i$-ésima clase que fue asignado al
      $r$-ésimo cluster $C_r$. La entropía del clustering se define
      como la suma ponderada de la entropía de cada cluster:

      $$E(\mathfrak{C}) = \sum_{r=1}^k \frac{n_r}{n} E(C_r)$$

      Un clustering perfecto tendrá clusters tal que cada cluster
      contenga documentos de una sola clase, en ese caso la entropía
      será 0. En general, conviene tener bajos valores de entropía.
\item \textbf{Pureza}: mide de qué forma cada cluster contiene documentos de
      una clase. La pureza de un cluster $C_r$ se define como

      $$P(C_r) = \frac{1}{n_r} max_i\{n^i_r\}$$

      la cual es la fracción del tamaño del cluster\ldots{}.\ldots{}.\ldots{}.\ldots{}.\ldots{}.\ldots{}.\ldots{}.\ldots{}.
\end{itemize}
\section{Identificación automática de eventos}
\label{sec-1.3}


   La identificación automática de eventos consiste en, dado un
   conjunto de documentos, donde cada documento está asociado a un
   evento (desconocido), es poder particionar el conjunto de
   documentos en clusters, de forma que cada cluster corresponda a
   todos los documentos asociados a un evento.

   Se considerará la definición de ``evento'' dada por \cite{Yang:1999:LAD:630307.630471}.

   \begin{defn} Un \emph{evento} es un suceso que ocurre en un período de tiempo
   determinado y en un lugar específico. \end{defn}

   El poder identificar eventos a partir de documentos publicados en
   los medios sociales permite mejorar la navegación de estos eventos,
   al mejorar la búsqueda tanto local como de motores de búsqueda.



\section{Resúmenes automáticos}
\label{sec-1.4}

\subsection{Evaluación de resúmenes}
\label{sec-1.4.1}

\section{Ranking de documentos}
\label{sec-1.5}


