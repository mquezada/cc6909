
\chapter{Conclusiones y trabajo a futuro}
\label{sec-5}

\label{cap:conclusiones}

Este trabajo consistió en el diseño e implementación de una
metodología de generación de resúmenes automáticos a partir de
contenido textual o multimedial de medios sociales. En particular:

\begin{enumerate}
\item Se mostró una metodología de obtención de datos a partir de eventos
   noticiosos y musicales, de forma de enriquecer los eventos con
   documentos sociales representados mediante tweets.
\item Utilizando estos datos, se generaron documentos usando una
   representación apropiada, independiente del contenido de éstos.
\item A partir de un conjunto de documentos, se identificaron los
   subtópicos de cada evento, mediante una solución de clustering.
\item Finalmente, desde los clusters se extrajo contenido relevante
   utilizando la información social de estos documentos, mediante una
   metodología simple para analizar casos de estudio.
\end{enumerate}

Este enfoque no ha sido estudiado aún muy en profundidad en el área de
investigación: el de generar resúmenes multimedia basándose en la
información social que mencionan los documentos.

%\section{Objetivos alcanzados y trabajo futuro}
%\label{sec-5.1}


Si bien los objetivos fueron cumplidos casi a cabalidad de acuerdo al
trabajo realizado, hay algunas componentes de la metodología propuesta
que pueden ser sujetas a mejoras, de forma de mejorar en criterios
como precisión, efectividad y eficiencia.

Como se mostró en el análisis de casos de estudio, hubo algunos casos
con buenos resultados y otros que son mejorables:

\begin{itemize}
\item Mejorar la extracción de términos de búsqueda para medios
  sociales. Es posible que una gran cantidad de términos no muy
  relacionados directamente entre sí obtengan muchos documentos sin
  relación entre ellos, tal como sucedió en el evento ``Stockholm''. O
  bien, diseñar una metodología distinta para la identificación de
  eventos y documentos asociados en medios sociales, considerando
  otros medios, como Facebook o Tumblr, abundantes en contenido
  multimedia.
\item El ranking de tweets muestra que en algunos casos, cuando los
  usuarios tienen muchos seguidores o un tweet sin importancia obtiene
  muchos retweets, estos tweets adquieren mucha relevancia dentro de
  un evento, nuevamente como ocurrió para ``Stockholm''. Una mejora a
  este aspecto puede considerar analizar el contenido del tweet o bien
  la relevancia del documento que menciona, si menciona alguno.
\item Se hace necesario realizar un análisis más fino para determinar el
  número de clusters adecuado a la hora de identificar los subtópicos
  de un evento, dado que los resultados pueden variar en calidad
  debido a este factor.
\end{itemize}
En cuanto a los objetivos:
\begin{itemize}
\item El objetivo principal fue cumplido casi completamente: se diseñó e
  implementó un sistema para generar resúmenes automáticos en base a
  documentos textuales y multimediales; sin embargo, sólo se
  implementó una metodología y no al nivel de un software de
  aplicación. Existen muchos aspectos que pueden ser mejorados aun.
\item De los objetivos específicos:

\begin{itemize}
\item El objetivo específico 1 fue cumplido. Se diseñó una metodología
    para extraer documentos relacionados a eventos en la Web,
    discutidos en las redes sociales online, en particular a Twitter.
\item El objetivo específico 2 fue cumplido. Se lograron identificar
    subtópicos y una metodología para determinar un número adecuado de
    éstos, basados en medidas internas del conjunto de documentos.
\item El objetivo específico 3 fue parcialmente cumplido. La estrategia
    utilizada para realizar ranking puede ser mejorada de forma de
    considerar no sólo características de la red de los mensajes
    publicados, sino también características de los documentos.
\item El objetivo específico 4 fue cumplido. Se analizaron 4 casos de
    pruebas, dos noticias y dos conciertos, y se evaluó la efectividad
    de la metodología. Se concluyó que existen ciertas características
    que influyen fuertemente sobre la calidad de los resultados, las
    cuales pueden ser corregidas o mejoradas.
\end{itemize}

\end{itemize}

La metodología implementada permite tener un punto de partida para el
desarrollo de un software para ser aplicado sobre eventos en redes
sociales y generar contenido multimedial basado en éstos. Los
resultados obtenidos muestran que la estrategia es adecuada para
lograr estos objetivos. El que se puedan realizar mejoras a la
metodología y a la implementación indica que la calidad de los
resultados puede ser mejorada, trayendo consigo documentos que
representen aún más los eventos que los discuten.