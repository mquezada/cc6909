
\chapter{Conclusiones}
\label{sec-5}

\label{cap:conclusiones}

Este trabajo consistió en el diseño e implementación de una
metodología de generación de resúmenes automáticos a partir de
contenido textual o multimedial de medios sociales. En particular:

\begin{enumerate}
\item Se mostró una metodología de obtención de datos a partir de eventos
   noticiosos y musicales, de forma de enriquecer los eventos con
   documentos sociales representados mediante tweets.
\item Utilizando estos datos, se generaron documentos usando una
   representación apropiada, independiente del contenido de éstos.
\item A partir de un conjunto de documentos, se identificaron los
   subtópicos de cada evento, mediante una solución de clustering.
\item Finalmente, desde los clusters se extrajo contenido relevante
   utilizando la información social de estos documentos, mediante una
   metodología simple para analizar casos de estudio.
\end{enumerate}
Este enfoque no ha sido estudiado aún muy en profundidad en el área de
investigación: el de generar resúmenes multimedia basándose en la
información social que mencionan los documentos.

\section{Objetivos alcanzados y trabajo futuro}
\label{sec-5.1}


Si bien los objetivos fueron cumplidos a cabalidad de acuerdo al
trabajo realizado, hay algunas componentes de la metodología propuesta
que pueden ser sujetas a mejoras, de forma de mejorar en criterios
como precisión, efectividad y eficiencia.

\section{Relevancia del trabajo realizado}
\label{sec-5.2}

