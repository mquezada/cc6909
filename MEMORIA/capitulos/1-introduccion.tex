\chapter{Introducción}
\label{sec-1}

Al igual que en el buffet de un restaurante, por mucho que se quisieran
comer todos los platos favoritos, es imposible comer todo lo que uno
quisiera por razones obvias. Una posibilidad es probar un poco de cada
comida, para luego saber qué es lo más delicioso y comer hasta
hartarse.

Pero, ¿qué hacer si hay demasiados platos y no se conocen todos? de
alguna manera hay que saber cuáles hay que probar, si el objetivo es
comer lo mejor posible. Un amigo puede recomendar una u otra comida,
lo cual puede servir para orientarse. Entonces se pueden escoger
pequeñas muestras de acuerdo a las recomendaciones.

Pasando a un contexto diferente, supóngase que este gran buffet es la
Web y los distintos platos corresponden a contenido publicado en
ella. Por lo tanto, dada la gran cantidad de información disponible,
se hace necesario poder encontrar lo más atractivo de acuerdo a la
preferencia del usuario o usuarios. Se está haciendo otra suposición
importante con esta analogía, y es que se está considerando que la
información es íntegramente para ser \emph{consumida}, y no, por ejemplo, para
generar más contenido, o para ser utilizada por máquinas, etc. Dentro de
este contexto se plantea la pregunta de cómo seleccionar el contenido
más atractivo dentro de todo lo que hay disponible en un momento dado.

Siguiendo el razonamiento de la analogía, una manera de poder
seleccionar sólo el contenido más ``atractivo'' (de acuerdo a las
preferencias del usuario), es probar un poco de cada uno. Diversos
esfuerzos han sido hechos para este propósito, entre ellos está el de
generar resúmenes automáticos\footnote{?? } a partir de uno o múltiples
documentos. Luego, utilizando las recomendaciones de otros usuarios es
posible ordenar estos resultados de acuerdo a la relevancia que éstos
les dan.

\section{Contexto}
\label{sec-1.1}

\section{Motivación}
\label{sec-1.2}

\section{Contribuciones}
\label{sec-1.3}

\section{Alternativas analizadas}
\label{sec-1.4}

\section{Objetivos}
\label{sec-1.5}

\subsection{Objetivo general}
\label{sec-1.5.1}

\subsection{Objetivos específicos}
\label{sec-1.5.2}

\section{Descripción general de la solución}
\label{sec-1.6}

\section{Resultados obtenidos}
\label{sec-1.7}


\begin{comment}
\lipsum[30-35]
\begin{enumerate}
	\item Item 1
	\begin{enumerate}
		\item Subitem 1
		\item Subitem 2 (ver Figura \ref{logofcfm})
	\end{enumerate}
	\item Item 2
	\item Item 3
\end{enumerate}
\begin{teo}
Se tiene que $$\int_0^t e^sds=e^t-1.$$
\end{teo}
\lipsum[36-40]
\end{comment}