
\chapter{Introducción}
\label{sec-1}


  Al igual que en el buffet de un restaurante, por mucho que se quisieran
  comer todos los platos, es imposible comer todo lo que uno
  quisiera por razones obvias. Una posibilidad es probar un poco de cada
  comida, para así saber qué es lo más delicioso y comer hasta
  hartarse.

  Pero, ¿qué hacer si hay demasiados platos y no se conocen todos? de
  alguna manera hay que saber cuáles hay que probar, si el objetivo es
  comer lo mejor posible. Un amigo, u otras personas que hayan comido
  en el restaurant pueden recomendar una u otra comida,
  lo cual puede servir para orientarse. Entonces se pueden escoger
  muestras de acuerdo a las recomendaciones.


  Pasando a un contexto diferente, supóngase que este gran buffet es la
  Web y los distintos platos corresponden a contenido publicado en
  ella. Por lo tanto, dada la gran cantidad de información disponible,
  se hace necesario poder encontrar lo más atractivo de acuerdo a la
  preferencia de los usuarios. Nótese que se está
  haciendo otra suposición importante con esta analogía, y es que se
  está considerando que la información es íntegramente para ser
  \emph{consumida}, y no, por ejemplo, para generar más contenido,
  conocimiento, o para ser utilizada por máquinas, etc. Dentro de
  este contexto se plantea la pregunta de cómo seleccionar el contenido
  más atractivo dentro de todo lo que hay disponible en un momento dado.

  Para hacerse la idea de cuál es la mejor oferta gastronómica de un
  restaurant, sin tener que probar todos los platos, se puede
  preguntar a la gente que come habitualmente ahí qué platos fueron
  los que más les han gustado, y así definir la calidad de éste y/o
  escoger los mejores platos. O bien, filtrar los documentos que
  hablan de un mismo evento de acuerdo a las preferencias de los
  usuarios, y luego presentarlo en la forma de un \emph{resumen} de cada
  uno.

  Este trabajo consistió en el desarrollo de una metodología
  que permite generar resúmenes automáticos de eventos determinados
  a partir de los documentos Web que hablan de éstos. No se hace
  ninguna suposición sobre el \emph{contenido} de estos documentos (pueden
  ser de texto, imágenes, videos, \emph{multimedios} en general). Los
  documentos del resumen son seleccionados de acuerdo a
  \emph{indicadores sociales}: los elementos con mayor impacto en las redes
  sociales son considerados más importantes.

  La metodología diseñada se implementó sobre dos tipos de eventos:
  noticias y conciertos musicales. Para obtenerlos, se utilizó
  el servicio de Google News\footnote{\href{http://news.google.com/}{http://news.google.com/} } y
  Last.fm\footnote{\href{http://last.fm/}{http://last.fm/} }. Los documentos y sus indicadores
  sociales se obtuvieron de la red social
  Twitter\footnote{\href{http://twitter.com/}{http://twitter.com/} }; y utilizando una representación
  adecuada de los documentos como vectores, se utilizó una estrategia
  de clustering para identificar los subtópicos de cada evento y así
  generar un resumen, basándose en los indicadores obtenidos.

  La estructura de este informe es como sigue: en este capítulo se
  comenta el contexto dentro del cual se desarrolló este sistema, las
  contribuciones realizadas, los objetivos y una descripción general
  de la solución; en Capítulo \ref{cap:antecedentes} se discute el estado del
  arte y el marco teórico del cual se desprende este trabajo; el
  Capítulo \ref{cap:problema} describe más en detalle el problema a resolver, su
  relevancia y sus dificultades, para luego, en el
  Capítulo \ref{cap:solucion}, describir la solución implementada, más
  un par de casos de estudio sobre los resultados obtenidos, para
  finalizar con las conclusiones de este trabajo en el Capítulo
  \ref{cap:conclusiones}.

\section{Contexto y Motivación}
\label{sec-1.1}


   La tasa de crecimiento de la cantidad de datos en la Web, y en
   particular, en las \emph{redes sociales online} (OSN, \emph{Online Social Networks}),
   es de tal magnitud que se vuelve necesario encontrar formas de
   filtrar y buscar sólo la información relevante dentro de todas las
   fuentes que hablan del mismo evento.

   Por otra parte, para poder entender lo que está pasando en el
   mundo a partir de los puntos de vista de toda las personas que
   manifiestan su opinión o los hechos recabados en las redes
   sociales, se hace necesario poder procesar toda esta
   información. Nuevamente, dado el gran volumen de datos, es
   necesario poder resumir esta información para ser rápidamente
   consumida.

   En el contexto de las redes sociales online, cada día los usuarios
   publican  millones de mensajes cortos con respecto a distintos
   tópicos, ya sean conversacionales, sobre eventos noticiosos,
   etc.\footnote{Pear Analytics. Twitter Study \href{http://es.scribd.com/doc/18548460/Pear-Analytics-Twitter-Study-August-2009}{http://es.scribd.com/doc/18548460/Pear-Analytics-Twitter-Study-August-2009} }
   Además, el auge de los teléfonos inteligentes o \emph{smartphones} con mayor
   capacidad de procesamiento e integrados con todo tipo de sensores
   (cámaras fotográficas, de vídeo, acelerómetro, osciloscopio, etc.),
   hace posible el generar aun más información y
   e incluso en tiempo real sobre lo que acontece en el mundo, en
   Internet, o bien sobre el estado particular de cada usuario.

   Este aumento y evolución de la generación de datos no sólo influye en la
   riqueza de éstos, sino también en el comportamiento de los usuarios
   a lo largo del tiempo. Actualmente,  una gran parte de éstos valora
   más el contenido de tipo multimedia (imágenes y videos)
   en las redes sociales online\footnote{The Rise of Visual Social Media \href{http://www.fastcompany.com/3000794/rise-visual-social-media}{http://www.fastcompany.com/3000794/rise-visual-social-media}. En el artículo se menciona un estudio sobre comportamiendo y preferencias de los usuarios en las redes sociales llevado a cabo por ROI Research: \href{http://www.slideshare.net/performics_us/performics-life-on-demand-2012-summary-deck}{http://www.slideshare.net/performics\_us/performics-life-on-demand-2012-summary-deck} }.
   Este contenido no sólo corresponde a documentos Web en el sentido
   tradicional, sino a objetos de más alto nivel como \emph{tweets}
   (mensajes cortos de la red social Twitter), imágenes (de servicios
   como Instagram\footnote{\href{http://instagr.am/}{http://instagr.am/} },
   Tumblr\footnote{\href{http://tumblr.com/}{http://tumblr.com/} }, etc.), vídeos
   (Youtube\footnote{\href{http://youtube.com/}{http://youtube.com/} }, Vimeo\footnote{\href{http://vimeo.com/}{http://vimeo.com/} }) e
   incluso sonidos (Soundcloud\footnote{\href{http://soundcloud.com/}{http://soundcloud.com/} }).
   Se hace entonces necesario encontrar formas para satisfacer estas
   necesidades de los usuarios, las cuales ya han sido
   abordadas en parte, como la generación de
   resúmenes automáticos orientado a motores de búsqueda, o la
   determinación de la relevancia tanto de documentos en la Web como de
   mensajes en las redes sociales.

   Surge como motivación el poder identificar y extraer contenido
   relevante de lo que está pasando en las redes sociales,
   clasificados dentro del concepto de eventos, y además avanzar un
   paso más arriba en el nivel de abstracción: considerar los
   documentos no por su contenido textual, lo que permite abarcar
   imágenes, vídeos, sonidos y multimedios en general. Algunas
   aplicaciones directas de esto son, entre otras:

\begin{itemize}
\item Ayudar al trabajo periodístico mediante una colección de
     contenido multimedia relacionado a un evento noticioso. Por
     ejemplo, la versión online de Radio
     Biobío\footnote{\href{http://www.biobiochile.cl/}{http://www.biobiochile.cl/} } frecuentemente publica
     breves artículos sobre sucesos que tienen impacto en las redes
     sociales, mostrando un pequeño conjunto de mensajes con
     comentarios de la gente\footnote{Como muestra: \href{http://www.biobiochile.cl/2012/12/01/aporte-de-lustrabotas-de-santiago-a-la-teleton-provoca-admiracion-en-redes-sociales.shtml}{http://www.biobiochile.cl/2012/12/01/aporte-de-lustrabotas-de-santiago-a-la-teleton-provoca-admiracion-en-redes-sociales.shtml}, y \href{http://www.biobiochile.cl/2012/12/01/rechazo-provocan-condicionamientos-de-compra-de-ripley-y-unimarc-para-donar-a-la-teleton.shtml}{http://www.biobiochile.cl/2012/12/01/rechazo-provocan-condicionamientos-de-compra-de-ripley-y-unimarc-para-donar-a-la-teleton.shtml} }.
     Una aplicación directa involucraría
     considerar además contenido multimedia, y organizar este
     contenido de acuerdo a la relevancia que tiene dentro de las
     redes.
\item Enriquecer la búsqueda en la Web a través de contenido
     multimedia. Una persona buscando información sobre un concierto
     podría obtener imágenes y vídeos de éste fácilmente una vez
     identificado el concierto.
\item Siguiendo lo anterior, un grupo musical podría obtener toda la
     información multimedia asociada a su concierto, tanto para sus
     fans como para ellos mismos, potenciando su popularidad.
\item Poder distinguir entre eventos similares rápidamente. Por
     ejemplo, un usuario que desee obtener información sobre ``Gaza'',
     puede referirse tanto a la banda de música como al conflicto en
     Israel. El poder distinguir rápidamente mediante una imagen o un
     vídeo acelera mucho el proceso. \emph{Una imagen vale más que mil palabras}.
\end{itemize}
   La metodología implementada es una primera aproximación que puede
   satisfacer los ejemplos mencionados.

\section{Objetivos}
\label{sec-1.2}

\subsection{Objetivo general}
\label{sec-1.2.1}


    El objetivo principal de este trabajo fue el siguiente:

    Diseñar e implementar un sistema que permita generar resúmenes
    automáticos de \emph{eventos}: información temporal publicada en redes
    sociales online sobre sucesos en particular. Esta información se
    basa en contenido textual y multimedial generado por los usuarios
    de estas redes, cuya relevancia se basa en el impacto generado en
    éstas.


\subsection{Objetivos específicos}
\label{sec-1.2.2}


\begin{enumerate}
\item Extraer datos relacionados a eventos en la Web, principalmente
       aquellos generados en redes sociales online. Estos datos pueden
       componerse tanto de información textual como multimedial.
\item Agrupar la información extraída de un evento en subtópicos.
\item Seleccionar los elementos más relevantes de cada subtópico para
       producir un resumen del evento.
\item Analizar la efectividad de la metodología propuesta sobre un
       conjunto de eventos noticiosos y conciertos.
\end{enumerate}
\section{Descripción general de la solución}
\label{sec-1.3}


   Para enfrentar el problema se diseñó una solución que consiste en
   tres componentes para resolverlo, resultando cada una en diferente
   grado de complejidad, siendo posible además ser mejoradas en el
   futuro. Para probar la viabilidad de la metodología, ésta fue
   aplicada sobre un conjunto de casos de prueba.

   En particular:

\begin{itemize}
\item Se llevó a cabo una metodología para la obtención de documentos y
     enriquecerlos con datos obtenidos de fuentes sociales;
\item Se diseñó un procedimiento que separar estos documentos en
     \emph{clusters}, \emph{sin considerar su contenido}. Sólo se utilizó la
     información social asociada; y
\item Se diseñó además un procedimiento para \emph{rankear} u ordenar los
     resultados de acuerdo a \emph{relevancia}, siendo ésta medida de
     acuerdo a la información social asociada a los documentos
     generados.
\end{itemize}
   Las componentes diseñadas fueron las siguientes:

\begin{enumerate}
\item La que obtiene descripciones de eventos a partir de fuentes de
      éstos en la Web, enriqueciéndolos con información social;
\item Otra componente que procesa y separa los documentos a partir de
      la información social; genera \emph{objetos Web} y los separa en
      subtópicos de cada evento, respectivamente; y
\item La componente que entrega los $k$ documentos más relevantes por
      cada evento obtenido, basándose en los subtópicos identificados.
\end{enumerate}