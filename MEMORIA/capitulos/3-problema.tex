
\chapter{Especificación del Problema}
\label{sec-3}

\label{cap:problema}

  El problema a resolver consistió en poder \emph{resumir} eventos en base
  al contenido textual y multimedial de la información publicada en
  medios sociales, tales como las redes sociales online. 

  Un resumen de un evento se consideró como una selección de elementos relevantes 
  y descriptivos del evento. Estos elementos corresponden a documentos u 
  \emph{objetos Web}: artículos, noticias, \emph{tweets}, fotografías, 
  vídeos, sonidos anotados, etc. Estos documentos deben ser descriptivos y poco redundantes:
  un pequeño grupo de documentos que permita describir el evento. Nótese que los
  objetos Web son objetos de alto nivel: no es sólo texto o contenido multimedia, sino
  que se consideran como objetos con información \emph{social} asociada, como la autoría
  del documento, tags o etiquetas, comentarios de consumidores (usuarios), viralidad en
  la Web, etc. Un resumen en este sentido no constituye una selección de datos de distintos documentos,
  sino que la selección íntegra de un documento que sea relevante y a su vez descriptivo del evento, filtrando
  o descartando los que no lo son.

  Se consideró un evento como una ocurrencia en el mundo real
  con un período de tiempo asociado y un conjunto de documentos, de
  volumen considerable, que discuten el acontecimiento y además son publicados
  dentro del período de tiempo asociado.

  El resumir un evento consiste en entregar un subconjunto de
  mensajes o de documentos mencionados en ellos que tengan
  relación directa con el evento en cuestión. Por ejemplo, el evento
  \texttt{Anef anuncia movilización nacional} contiene muchos mensajes, tales
  como:

\begin{itemize}
\item \texttt{RT @econtingencia: Reajuste: Trabajadores del sector público}
    \texttt{convocan a movilizaciones: La negociación entra a su recta final:}
    \texttt{El l... h ...}
\item \texttt{CUT de La Araucanía se adherirá a marcha nacional en rechazo a}
    \texttt{reajuste salarial: La Central Unitaria de}
    \texttt{Trabaja... http://t.co/IcgohlTR}
\item \texttt{Reajuste: Trabajadores del sector público convocan a}\\
    \texttt{movilizaciones http://t.co/pWC7q6NI}
\item \texttt{RT @biobio: CUT de La Araucanía se adherirá a marcha nacional en}
    \texttt{rechazo a reajuste salarial ofrecido por el Gobierno}
    \texttt{http://t.co/IOieDido}
\item \texttt{http://t.co/ttFp0XgH Trabajadores del sector público anuncian}
    \texttt{movilizaciones}
\item \texttt{RT @DanielaLopezLv: http://t.co/ttFp0XgH Trabajadores del sector}
    \texttt{público anuncian movilizaciones}
\item \texttt{RT @Barbara\_figue: Reajuste: Trabajadores del sector público}
    \texttt{convocan a movilizaciones http://t.co/DeXKACSR vía @nacioncl}
\end{itemize}

  El resumen consistirá en una colección de objetos, documentos o
  mensajes, donde en este caso los documentos están representados por
  las URLs contenidas en los mensajes, de forma que esta colección
  represente lo mejor posible el evento, siendo de tamaño
  considerablemente menor al total de mensajes/documentos del evento;
  garantizando además baja redundancia entre ellos.

  El problema de identificar contenido multimedial en medios sociales
  a partir de eventos de por sí es un problema desafiante dada la
  heterogeneidad y la naturaleza ruidosa de los datos: los mensajes
  son breves, y pueden contener errores gramaticales o de ortografía;
  además, pueden ser ambiguos respecto al evento que están haciendo
  referencia (por ejemplo, un mensaje que mencione la palabra ``Gaza''
  puede referirse al conflicto en medio oriente o bien a la banda
  musical ``Gaza'').

  Este problema no ha sido abordado en profundidad con anterioridad,
  si se considera el aspecto social para la medida de relevancia,
  sino que los esfuerzos se han concentrado principalmente en la
  generación de resúmenes textuales por una parte\cite{Conrad:2005:EDC:1165485.1165513,allan2002topic,DBLP:conf:spire:Bravo-MarquezM12,Diakopoulos:2012:FAS:2208276.2208409},
  y en la recolección de contenido multimedia para eventos por
  otra\cite{Becker:2012:ICP:2124295.2124360,Liu:2011:USM:2072609.2072613,Becker:2010:LSM:1718487.1718524}. Sin embargo, aún no
  han habido muchos esfuerzos en cuanto a utilizar la información
  social asociada a los documentos para generar medidas de relevancia
  en la generación de resúmenes visuales automáticos.

  Algunas aplicaciones de una solución al problema planteado son:

\begin{itemize}
\item Poder distinguir entre dos eventos rápidamente, por ejemplo, si
    ``Gaza'' se refiere al evento musical o al conflicto en medio
    oriente.
\item Comprender la información contenida en grandes fuentes de datos
    rápidamente mediante un resumen visual.
\item Mejorar la calidad del servicio de distintos rubros, al conocer de
    manera más eficaz el feedback de los usuarios (por ejemplo, un
    \emph{review} en vídeo del tablet Nexus 7 de Google, o fotografías de
    baterías de teléfonos móviles que explotan sin razón aparente).
\item Tener información más completa sobre eventos musicales, por
    ejemplo, una persona que quiera asistir a un festival de música
    puede ver rápidamente fotografías y vídeos del evento realizado en
    otras partes, para poder tomar una mejor decisión.
\item Mejorar el trabajo periodístico, al tener mejor cobertura de
    eventos masivos como manifestaciones o eventos políticos como
    elecciones o discursos, entre otros.
\end{itemize}