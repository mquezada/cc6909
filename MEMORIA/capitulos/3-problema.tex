\chapter{Especificación del Problema}
\label{sec-3}


  El problema a resolver consiste en poder \emph{resumir} eventos en base
  al contenido textual y multimedial de la información publicada en
  medios sociales, tales como las redes sociales online.

  Se considerará un evento como una ocurrencia en el mundo real
  con un período de tiempo asociado y un conjunto de mensajes, de
  volumen considerable, que discutan la ocurrencia y además publicados
  dentro del período de tiempo.

  El resumir un evento consiste en entregar un subconjunto de
  mensajes o de documentos mencionados en ellos que tengan
  relación directa con el evento en cuestión. Por ejemplo, el evento
  \texttt{Anef anuncia movilización nacional} contiene muchos mensajes, tales
  como:
\begin{itemize}
\item \texttt{RT @econtingencia: Reajuste: Trabajadores del sector público}
    \texttt{convocan a movilizaciones: La negociación entra a su recta final:}
    \texttt{El l... h ...}
\item \texttt{CUT de La Araucanía se adherirá a marcha nacional en rechazo a}
    \texttt{reajuste salarial: La Central Unitaria de}
    \texttt{Trabaja... http://t.co/IcgohlTR}
\item \texttt{Reajuste: Trabajadores del sector público convocan a}
    \texttt{movilizaciones http://t.co/pWC7q6NI}
\item \texttt{RT @biobio: CUT de La Araucanía se adherirá a marcha nacional en}
    \texttt{rechazo a reajuste salarial ofrecido por el Gobierno}
    \texttt{http://t.co/IOieDido}
\end{itemize}
  El resumen consistirá en una colección de objetos, los cuales pueden
  ser o bien mensajes o bien algunos de los documentos mencionados en
  ellos (en este caso representados por las URLs contenidas en los
  mensajes), de forma que esta colección sea de tamaño mucho menor que
  el total de mensajes, y además que los elementos de esta colección
  representen de alguna forma todos los puntos de vista o aspectos
  importantes del evento.

  Se deben tomar en consideración las características de estos
  mensajes: muchos de ellos tendrán errores de ortografía, redacción,
  gramática, o semántica (que hablen de otro tema). Esto se traduce
  como el ruido de los datos. Una solución a este problema debe tomar
  en consideración el ruido intrínseco de los datos. Además, estos
  mensajes son breves y no expresan en profunidad un punto de vista
  del evento, sino que usualmente pueden tener una o dos
  características cruciales que permitan distinguirlos de entre todos
  los que existen, por ejemplo, una mención a una imagen o a un
  vídeo que explique una parte del evento.
