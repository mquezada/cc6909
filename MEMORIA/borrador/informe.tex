% Created 2012-11-24 Sat 19:51
\documentclass[upright, contnum]{umemoria}
\usepackage[latin1]{inputenc}
\inputencoding{latin1}
%\usepackage[T1]{fontenc}
%\usepackage{fontspec}
\usepackage{graphicx}
%\defaultfontfeatures{Mapping=tex-text}
%\setmainfont{Linux Libertine O}
%\setmonofont[Scale=0.8]{DejaVu Sans Mono}

\let\evensidemargin\oddsidemargin
\reversemarginpar

\pagestyle{empty}

\depto{Ciencias de la Computación}
\carrera{Ingeniero Civil en Computación}
\comision{Sergio Ochoa Delorenzi}{Mauricio Marín Caihuan}{}
\guia{Bárbara Poblete Labra}


\title{IDENTIFICACIÓN DE CONTENIDO MULTIMEDIA RELEVANTE A PARTIR DE EVENTOS UTILIZANDO SU INFORMACIÓN SOCIAL}
\author{MAURICIO DANIEL QUEZADA VEAS}
\date{DICIEMBRE 2012}

\begin{document}

\maketitle





\begin{abstract}
asdf
\end{abstract}

\begin{dedicatoria}
Jason Funk disipa patitos
\end{dedicatoria}

\begin{thanks}
asdf
\end{thanks}

\cleardoublepage
\tableofcontents
%\cleardoublepage
%\listoftables
%\cleardoublepage
%\listoffigures

\mainmatter


\chapter{Introducción}
\label{sec-1}


Al igual que en el buffet de un restaurante, por mucho que se quisieran
comer todos los platos favoritos, es imposible comer todo lo que uno
quisiera por razones obvias. Una posibilidad es probar un poco de cada
comida, para luego saber qué es lo más delicioso y comer hasta
hartarse.

Pero, ¿qué hacer si hay demasiados platos y no se conocen todos? de
alguna manera hay que saber cuáles hay que probar, si el objetivo es
comer lo mejor posible. Un amigo puede recomendar una u otra comida,
lo cual puede servir para orientarse. Entonces se pueden escoger
pequeñas muestras de acuerdo a las recomendaciones.

Pasando a un contexto diferente, supóngase que este gran buffet es la
Web y los distintos platos corresponden a contenido publicado en
ella. Por lo tanto, dada la gran cantidad de información disponible,
se hace necesario poder encontrar lo más atractivo de acuerdo a la
preferencia del usuario o usuarios. Se está haciendo otra suposición
importante con esta analogía, y es que se está considerando que la
información es íntegramente para ser \emph{consumida}, y no, por ejemplo, para
generar más contenido, o para ser utilizada por máquinas, etc. Dentro de
este contexto se plantea la pregunta de cómo seleccionar el contenido
más atractivo dentro de todo lo que hay disponible en un momento dado.

Siguiendo el razonamiento de la analogía, una manera de poder
seleccionar sólo el contenido más ``atractivo'' (de acuerdo a las
preferencias del usuario), es probar un poco de cada uno. Diversos
esfuerzos han sido hechos para este propósito, entre ellos está el de
generar resúmenes automáticos\footnote{?? } a partir de uno o múltiples
documentos. Luego, utilizando las recomendaciones de otros usuarios es
posible ordenar estos resultados de acuerdo a la relevancia que éstos
les dan.

\section{Contexto}
\label{sec-1.1}

\section{Motivación}
\label{sec-1.2}

\section{Contribuciones}
\label{sec-1.3}

\section{Alternativas analizadas}
\label{sec-1.4}

\section{Objetivos}
\label{sec-1.5}

\subsection{Objetivo general}
\label{sec-1.5.1}

\subsection{Objetivos específicos}
\label{sec-1.5.2}

\section{Descripción general de la solución}
\label{sec-1.6}

\section{Resultados obtenidos}
\label{sec-1.7}


\chapter{Antecedentes}
\label{sec-2}

\section{Conceptos involucrados}
\label{sec-2.1}

\section{Soluciones existentes}
\label{sec-2.2}


\chapter{Especificación del Problema}
\label{sec-3}

\section{Descripción detallada}
\label{sec-3.1}

\section{Relevancia de una solución}
\label{sec-3.2}

\section{Características de calidad}
\label{sec-3.3}

\section{Criterios de aceptación}
\label{sec-3.4}


\chapter{Descripción de la Solución}
\label{sec-4}

\section{Desafíos técnicos}
\label{sec-4.1}

\section{Metología de desarrollo}
\label{sec-4.2}

\section{Casos de estudio}
\label{sec-4.3}

\section{Validación}
\label{sec-4.4}

\chapter{Conclusiones}
\label{sec-5}

\section{Resumen del trabajo realizado}
\label{sec-5.1}

\section{Objetivos alcanzados}
\label{sec-5.2}

\section{Relevancia del trabajo realizado}
\label{sec-5.3}

\section{Trabajo futuro}
\label{sec-5.4}




\chapter{Unclassified D:}
\label{sec-6}

  

\section{Twitter}
\label{sec-6.1}

Twitter es un servicio que permite conectar a personas mediante
mensajes cortos, rápidos y frecuentes. Estos mensajes son publicados
en el perfil del usuario que los emite, pueden ser vistos directamente
por los seguidores de este usuario o ser vistos directamente en el
perfil o buscándolos mediante una funcionalidad que provee el
servicio. Además, un usuario puede \emph{seguir} a otros para poder ver en
su \emph{timeline} los mensajes de todos a quienes sigue.


\section{Metodología de obtencion del dataset}
\label{sec-6.2}


Se describe a continuación el proceso diseñado para la obtención de
datos para alimentar al sistema implementado.

Las etapas de generación del Dataset son las siguientes:

\begin{itemize}
\item Recolección de eventos (noticias y conciertos);
\item Enriquecimiento de los eventos existentes mediante tweets; e
\item Identificación de documentos a partir de los tweets por cada evento.
\end{itemize}
Se recolectaron datos (eventos y tweets) desde el 19 de noviembre de
2012 hasta XXXXXXXXXXXX todos los días desde la medianoche hasta que
el procedimiento terminaba exitosamente.

\subsection{Recolección de eventos}
\label{sec-6.2.1}


Se consideraron dos tipos de eventos para el sistema: noticias y
conciertos musicales. Los conciertos incluyen festivales de varios
artistas.

\begin{itemize}
\item Noticias
  Para obtener las noticias, se utilizó el servicio de Google
  News\footnote{\href{http://news.google.com}{http://news.google.com} }. Existe una API (en proceso de
  obsolescencia, pero funcional a la fecha de este trabajo) que permite
  obtener no sólo los titulares y breve descripción de cada noticia,
  sino también un conjunto de entre 4-10 noticias relacionadas de otras
  fuentes. Esto sirvió para alimentar los términos de búsqueda para la
  etapa siguiente. Se guardaron los siguientes datos de una noticia:

\begin{itemize}
\item Título,
\item Descripción,
\item URL de la fuente, y
\item Titulares de las noticias relacionadas.
\end{itemize}

\item Conciertos
  Utilizando el servicio de Last.fm para obtener los conciertos y
  festivales de una ubicación en
  particular\footnote{\href{http://www.lastfm.es/api/show/geo.getEvents}{http://www.lastfm.es/api/show/geo.getEvents} }, se
  obtuvieron los conciertos y festivales de las siguientes
  ubicaciones:

\begin{itemize}
\item Santiago, Chile;
\item Londres, Inglaterra;
\item Glastonbury, Inglaterra;
\item Las Vegas, Nevada, EE.UU.; y
\item Estocolmo, Suecia.
\end{itemize}

\item Título del evento (concierto o festival);
\item Artistas que participan; y
\item Fechas de inicio y término (esta última no siempre está como
    dato).

  Además de otros datos descriptivos, como la ubicación, descripción
  breve, sitio web de la banda o festival, etc.
\end{itemize}
Cada vez que se obtienen los eventos se vuelven a obtener los
conciertos, pero sólo agregando los nuevos. Las noticias siempre son
nuevas, aun así por implementación no se consideraron los repetidos.
  
\subsection{Enriquecimiento de eventos}
\label{sec-6.2.2}


Se obtuvieron tweets utilizando el servicio de búsqueda que provee
Twitter en su
API\footnote{\href{https://dev.twitter.com/docs/api/1.1/get/search/tweets}{https://dev.twitter.com/docs/api/1.1/get/search/tweets} }. El
objetivo es enriquecer los eventos con la información social que hay
en la Web sobre éstos. 

Para cada uno de los eventos obtenidos en la fase anterior, se
utilizaron los términos de búsqueda asociados a ellos: los titulares
de las noticias relacionadas y los nombres de los artistas para los
eventos noticiosos y musicales, respectivamente.

\begin{itemize}
\item Para las noticias, se hace una búsqueda en Twitter de los titulares
  al mismo tiempo en que se obtienen de Google News, y nuevamente al
  día siguiente, es decir, 2 búsquedas por cada titular de un evento.
  Se quitan las tildes y caracteres no ASCII y las stopwords, para
  evitar problemas con la implementación y no hacer calce de stopwords
  en la búsqueda de Twitter, respectivamente.
\item Para los conciertos y festivales, se utilizaron los nombres de los
  artistas y del evento como términos de búsqueda. De acuerdo a la
  información asociada al evento, se busca por una mayor cantidad de
  días:

\begin{itemize}
\item Se busca desde un día antes de inicio del evento;
\item Si está presente la fecha de término del evento, se busca cada día
    dentro del intervalo ``fecha de inicio'' a ``fecha de término'' hasta
    tres días terminado el evento.
\item Si no está presente la fecha de término (por ejemplo, un concierto
    o un festival de un día), se busca hasta tres días pasada la fecha
    de inicio.
\end{itemize}

\end{itemize}
\subsection{Identificación de documentos a partir de tweets}
\label{sec-6.2.3}


    Luego de obtener los tweets asociados a cada evento, el siguiente
    paso fue generar los documentos que fueron usados para la
    generación de los resúmenes. Nuevamente, el modelo consistió en que cada
    documento se modeló como un vector de palabras, donde el
    identificador del documento es una URL, y sus componentes
    corresponden al contenido de los tweets que tienen esa URL en el
    texto del mensaje.

    El caso en el que un tweet no tenía ninguna URL en su contenido
    fue abordado de la siguiente forma: la URL asociada es una tal que
    representa al mismo tweet (utilizando el servicio de Twitter), y
    el contenido de ese documento es el mismo tweet, de forma de no
    dejar el tweet sin ser representado.

    Este proceso fue abordado recorriendo todos los eventos del
    dataset, observando todos los tweets asociados a cada evento,
    extrayendo la URL si es que hay alguna y guardando el documento
    con el nuevo tweet. Se marcan los tweets observados para no tener
    que repetir el proceso, ya que es intensivo en conexión a la red.

    Dada la condición breve de los mensajes publicados en la red
    social, muchos de los usuarios y/o servicios que publican mensajes
    con una URL n su interior suelen utilizar \emph{acortadores} (\emph{url shorteners})
    para los enlaces, y así no utilizar mucho espacio dentro de un
    mensaje. Otra ventaja que ofrecen es que algunos servicios como
    \hyperref[sec-6.2.3]{bit.ly} dan estadísticas sobre los visitantes a estos enlaces (y
    así saber quiénes vienen de cierta red social u otra, por
    ejemplo). Twitter, a su vez, actualmente también ofrece
    acortamiento de URLs por defecto. Esto suele producir que un enlace
    acortado se resuelva a otro enlace también acortado, por lo que es
    necesario resolver la URL completa para evitar duplicados o
    \emph{pseudo-duplicados} (en el caso en que dos URLs sintácticamente
    distintas apunten al mismo recurso). EN LA FIGURA\ldots{}...

    FIGURA DE LINKS CORTOS

    Por lo anterior, una vez identificada la URL del texto de un
    tweet, se resuelve su URL completa (que puede ya serlo de
    antemano), lo que consume recursos de ancho de banda y
    tiempo. 

\section{Performance}
\label{sec-6.3}


Tiempo? espacio? por evento?

\section{Restricciones de la API de Twitter}
\label{sec-6.4}


   La API de búsqueda de Twitter permite obtener tweets de acuerdo a un
   término de búsqueda. Se utilizó este servicio para enriquecer los
   eventos con información social utilizando como términos de búsqueda
   tanto los títulos de las noticias como los nombres de los artistas
   para las noticias y los conciertos, respectivamente. 
   
   Funciona de la siguiente forma: cada vez que se hace un request a la
   URL dada por el servicio, éste retorna a lo más 100 tweets por página, con un
   máximo de 15 páginas (indicando en el request qué página queremos
   consultar), dando como total hasta 1500 tweets por búsqueda. Existirán
   términos de búsqueda que no presenten ningún resultado  (ya sea por
   estar mal escritos o simplemente que no sean un tópico de discusión), o por
   el contrario, que se generen más tweets que los retornados por la
   búsqueda por cada ventana de tiempo que demore ésta (por ejemplo, un
   \emph{trending topic} o tópico que sea muy mencionado en la red social).
   
   Existe una limitación de uso de este servicio: sólo es posible hacer
   hasta 180 requests por cada 15 minutos, o 1 request cada 5
   segundos. Además, sólo retorna tweets de hasta 7 días de antigüedad, y
   sus resultados no son necesariamente en tiempo real y su estabilidad
   varía de acuerdo a factores externos.
   
   Los tweets retornados vienen en formato \texttt{JSON} (\emph{Javascript Simple Object Notation}),
   e incluyen varios metadatos sobre el tweet aparte de los principales,
   como autor, fecha, contenido. Algunos de estos metadatos son:
   
\begin{itemize}
\item Cantidad de \emph{retweets} hechos hasta la fecha;
\item Si posee alguna URL o \emph{hashtag} en el texto;
\item Si es una \emph{mención} a otro usuario;
\item La ubicación de donde se envió el tweet;
\item etc.
\end{itemize}
  Además incluye datos sobre el autor, como por ejemplo:

\begin{itemize}
\item Si la cuenta está \emph{verificada};
\item La cantidad de seguidores del usuario;
\item Cantidad de amigos (seguidores que también lo siguen);
\item Cantidad de tweets;
\item Su descripción, y si incluye alguna URL, etc;
\item Ubicación (dada por el mismo usuario);
\item Fecha de creación de la cuenta;
\item etc.
\end{itemize}
\nocite{*}
\bibliographystyle{plain}
\bibliography{bibliografia}






\end{document}
